\documentclass{article}

\usepackage[vmarginratio=1:1,a4paper,body={6.5in, 9.5in}]{geometry}

\usepackage[T2A]{fontenc}
\usepackage[utf8]{inputenc}
\usepackage[russian]{babel}

\usepackage{color}

\usepackage{amsmath}
\DeclareMathOperator*{\argmin}{arg\,min}

\usepackage{amssymb}
\usepackage{amsfonts}
\usepackage{amsthm}
\newtheorem{theorem}{Theorem}
\newtheorem{definition}{Definition}

\usepackage{hyperref}
\usepackage{graphicx}
\graphicspath{{./img/}}


\begin{document}
\section{Is probability theory well-founded or do you believe in the theory of evolution?}

I’ll start with the least squares methods through the maximum likelihood estimation;
this requires some (at least superficial) knowledge of probability theory. 

I was once asked if I believed in evolutionary theory. 
Take a short break, think about how you would answer.
Being puzzled by the question, I have answered that I find it plausible. 
Scientific theory has little to do with faith.
In short, a theory only builds a model of the world, and there is no need to believe in it.
Moreover, the Popperian criterion\cite{} requires a scientific theory be able to be falsifiable. 
A solid theory must possess, first of all, the power of prediction.
For example, if you genetically modify crops in such a way that they produce pesticides themselves, 
it is only logical that pesticide-resistant insects would appear. 
However, it is much less obvious that this process can be slowed down by growing regular plants side by side with genetically modified plants. 
Based on evolutionary theory, the corresponding modelling has made this prediction\cite{}, and it seems to have been validated\cite{}.


$$P(a) = .4  \qquad  P(a\wedge b) = 0 \qquad P(b) = .3 \qquad P(a\vee b) = .8$$
	
\begin{tabular}{cccccc}
	\multicolumn{2}{c}{Bets of the Agent 2}  &  \multicolumn{4}{c}{Result for the Agent 1} \\
	\hline
	{\tiny Event} & {\tiny Bet amount} & {\tiny $a\wedge b$} & {\tiny $a\wedge \neg b$} &  {\tiny $\neg a\wedge b$} &  {\tiny $\neg a\wedge\neg b$} \\
	\hline
	$a$             & 4-6 & -6 & -6 &  4 &  4 \\
	$b$             & 3-7 & -7 &  3 & -7 &  3 \\
	$\neg(a\vee b)$ & 2-8 &  2 &  2 &  2 & -8 \\
	\hline
	&     &-11 & -1 & -1 & -1
\end{tabular}
	
\begin{itemize}
	\item $0\leq P(a)\leq 1$
	\item $P(true)=1$, $P(false) = 0$
	\item $P(a\vee b) = P(a) + P(b) - P(a\wedge b)$
\end{itemize}


$$
P(k;n,p) = C_n^k p^k (1-p)^{n-k}
$$

$$P(k) = C_{10}^k \frac{1}{2^k}\left(1-\frac{1}{2}\right)^{n-k} = \frac{C_{10}^k}{2^n}$$

$$\mathcal{L}(p) = C_{10}^7 p^7 (1-p)^3$$

$$\log \mathcal{L}(p) = \log C_{10}^7 + 7 \log p + 3\log (1-p)$$

$$\frac{d \log \mathcal{L}}{dp} = 0$$

$$\frac{d \log \mathcal{L}}{dp} = \frac{7}{p} - \frac{3}{1-p} = 0$$

$p=0.7$

$$\frac{d^2 \log \mathcal{L}}{dp^2} = -\frac{7}{p^2} - \frac{3}{(1-p)^2}$$

$$\frac{d^2 \log \mathcal{L}}{dp^2}(0.7)  \approx -48 < 0$$

$$
\{U_j\}_{j=1}^{N}
$$

$$
p(U_j) = \frac{1}{\sqrt{2\pi}\sigma} \exp\left(-\frac{(U_j-U)^2}{2\sigma^2}\right)
$$


\begin{align*}
\log \mathcal{L}(U,\sigma) & = \log \left(\prod\limits_{j=1}^N  \frac{1}{\sqrt{2\pi}\sigma} \exp\left(-\frac{(U_j-U)^2}{2\sigma^2}\right)\right) =\\
& = \sum\limits_{j=1}^N \log \left(\frac{1}{\sqrt{2\pi}\sigma} \exp\left(-\frac{(U_j-U)^2}{2\sigma^2}\right)\right) = \\
& = \sum\limits_{j=1}^N \left(\log \left(\frac{1}{\sqrt{2\pi}\sigma}\right) -\frac{(U_j-U)^2}{2\sigma^2}\right) = \\
& = -N \left(\log\sqrt{2\pi} + \log\sigma\right) - \frac{1}{2\sigma^2} \sum\limits_{j=1}^N (U_j-U)^2
\end{align*}


$$
\frac{\partial\log\mathcal{L}}{\partial U}    =  \frac{1}{\sigma^2}\sum\limits_{j=1}^N (U_j-U) = 0 
$$

$$
U = \frac{\sum\limits_{j=1}^N U_j}{N}
$$

$$
\frac{\partial\log\mathcal{L}}{\partial\sigma} =  -\frac{N}{\sigma} + \frac{1}{\sigma^3}\sum\limits_{j=1}^N (U_j-U)^2 = 0
$$

$$
\sigma = \sqrt{\frac{\sum\limits_{j=1}^N (U_j-U)^2}{N}} 
$$

$$
\{I_j, U_j\}_{j=1}^{N}
$$

\begin{align*}
\log \mathcal{L}(R,\sigma) & = \log \left(\prod\limits_{j=1}^N  \frac{1}{\sqrt{2\pi}\sigma} \exp\left(-\frac{(U_j-R I_j)^2}{2\sigma^2}\right)\right) =\\
& = -N \left(\log\sqrt{2\pi} + \log\sigma\right) - \frac{1}{2\sigma^2} \sum\limits_{j=1}^N (U_j- R I_j)^2
\end{align*}

\begin{align*}
\frac{\partial\log\mathcal{L}}{\partial R} &=  -\frac{1}{2\sigma^2}\sum\limits_{j=1}^N -2I_j (U_j- R I_j) = \\
&= \frac{1}{\sigma^2}\left(\sum\limits_{j=1}^N I_jU_j - R\sum\limits_{j=1}^N I_j^2\right) = 0
\end{align*}

$$
R = \frac{\sum\limits_{j=1}^N I_jU_j}{\sum\limits_{j=1}^N I_j^2}
$$

$$
\{x_j, y_j\}_{j=1}^{N}
$$

\begin{align*}
\log \mathcal{L}(a, b,\sigma) & = \log \left(\prod\limits_{j=1}^N  \frac{1}{\sqrt{2\pi}\sigma} \exp\left(-\frac{(y_j - a x_j - b)^2}{2\sigma^2}\right)\right) =\\
& = -N \left(\log\sqrt{2\pi} + \log\sigma\right) - \frac{1}{2\sigma^2} \sum\limits_{j=1}^N (y_j- a x_j - b)^2
\end{align*}


$$
S(a, b) = \sum\limits_{j=1}^N (y_j- a x_j - b)^2
$$


\begin{align*}
\frac{\partial S}{\partial a} &= \sum\limits_{j=1}^N 2 x_j (a x_j + b - y_j) = 0 \\
\frac{\partial S}{\partial b} &= \sum\limits_{j=1}^N 2 (a x_j + b - y_j) = 0
\end{align*}

$$
\left \{ \begin{array}{r l}
a \sum\limits_{j=1}^N x_j^2 + b \sum\limits_{j=1}^N x_j  & = \sum\limits_{j=1}^N x_j y_j\\
a \sum\limits_{j=1}^N x_j   + b N                        & = \sum\limits_{j=1}^N y_j
\end{array} \right.
$$

\begin{align*}
a &= \frac{N \sum\limits_{j=1}^N x_j y_j - \sum\limits_{j=1}^N x_j \sum\limits_{j=1}^N y_j}{N\sum\limits_{j=1}^N x_j^2 - \left(\sum\limits_{j=1}^N x_j\right)^2} \\
b &= \frac{1}{N}\left(  \sum\limits_{j=1}^N y_j - a  \sum\limits_{j=1}^N x_j \right)
\end{align*}




\end{document}
